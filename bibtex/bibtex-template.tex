\documentclass[a4paper,11pt]{article}

% Codificación
\usepackage[utf8]{inputenc}

% Idioma
\usepackage[spanish]{babel}
\selectlanguage{spanish}

% Hay que pelearse con babel-spanish para el alineamiento del punto decimal
\decimalpoint
\usepackage{dcolumn}
\newcolumntype{d}[1]{D{.}{\esperiod}{#1}}
\makeatletter
\addto\shorthandsspanish{\let\esperiod\es@period@code}
\makeatother

% To work with images
\usepackage{float}
\usepackage{graphicx}

% To work with bibtex
\usepackage[backend=bibtex,style=numeric,sorting=none]{biblatex}
\bibliography{references}
% Coloured links	
\usepackage{hyperref}
\hypersetup{
  colorlinks   = true, %Colours links instead of ugly boxes
  urlcolor     = blue, %Colour for external hyperlinks
  linkcolor    = blue, %Colour of internal links
  citecolor   = red %Colour of citations
}

\begin{document}

%%%%%%%%%%%%%%%%%%%%%%%%%%%%%%%%%%%%%%%%%
% University Assignment Title Page 
% LaTeX Template
% Version 1.0 (27/12/12)
%
% This template has been downloaded from:
% http://www.LaTeXTemplates.com
%
% Original author:
% WikiBooks (http://en.wikibooks.org/wiki/LaTeX/Title_Creation)
% Modified by: NCordon (https://github.com/NCordon)
%
% License:
% CC BY-NC-SA 3.0 (http://creativecommons.org/licenses/by-nc-sa/3.0/)
% 
% Instructions for using this template:
% This title page is capable of being compiled as is. This is not useful for 
% including it in another document. To do this, you have two options: 
%
% 1) Copy/paste everything between \begin{document} and \end{document} 
% starting at \begin{titlepage} and paste this into another LaTeX file where you 
% want your title page.
% OR
% 2) Remove everything outside the \begin{titlepage} and \end{titlepage} and 
% move this file to the same directory as the LaTeX file you wish to add it to. 
% Then add \input{./title_page_1.tex} to your LaTeX file where you want your
% title page.
%
%%%%%%%%%%%%%%%%%%%%%%%%%%%%%%%%%%%%%%%%%
  \begin{titlepage}

  \newcommand{\HRule}{\rule{\linewidth}{0.5mm}} % Defines a new command for the horizontal lines, change thickness here

  \center % Center everything on the page
  
  %----------------------------------------------------------------------------------------
  %	HEADING SECTIONS
  %----------------------------------------------------------------------------------------
  \textsc{\LARGE Universidad de Granada}\\[1.5cm]
  \textsc{\Large Subtitle}\\[0.5cm] 

  %----------------------------------------------------------------------------------------
  %	TITLE SECTION
  %----------------------------------------------------------------------------------------
  \bigskip
  \HRule \\[0.4cm]
  { \huge \bfseries Title}\\[0.4cm] % Title of your document
  \HRule \\[1.5cm]
  
  %----------------------------------------------------------------------------------------
  %	AUTHOR SECTION
  %----------------------------------------------------------------------------------------

  \begin{minipage}{0.4\textwidth}
  \begin{center} \large
  \emph{Fulano Mengano Perengano}\\
  \end{center}
  \end{minipage}

  %----------------------------------------------------------------------------------------
  %	LOGO SECTION
  %----------------------------------------------------------------------------------------

  \begin{center}
  %  \includegraphics[width=9cm]{ugr.jpg}
  \end{center}
  %----------------------------------------------------------------------------------------

  \vspace{\fill}% Fill the rest of the page with whitespace
  \large\today
  \end{titlepage}  

  \newpage
  \tableofcontents
  \newpage
    \section{Sección de prueba}
      Estoy citando a \cite{prueba}
  \printbibliography
\end{document}